\newif\ifToDo\ToDotrue
\newif\ifSpec\Spectrue
% To be set before the macro package since it stores these values.
% Letter: 216 x 279, A4: 210 x 297
\advance\hsize by-6mm\advance\vsize by18mm
\input epsf
\input btxmac
\input BookMacros/BookMacros
\input xref
\def\EjectLevel{1}% no page break for chapters
%\def\LaTeX{L\kern-.36em\raise.3ex\hbox{\sc \uppercasesc a}\kern-.15em\TeX}%
\def\LaTeX{L\kern-.36em\raise.3ex\hbox{a}\kern-.15em\TeX}%
\def\RmNTS#1{{\rm$\langle$#1$\rangle$}} % nonterminal symbol
\def\SlNTS#1{\RmNTS{{\sl#1\/}}} % nonterminal symbol
\centerline{\seventeenss Plain \TeX\ BookMacros}
\vskip2ex
\centerline{\twelvess Version \Version}
\Toc\Lof\Lot
\Chapt{Introduction}
Typesetting a book document with Knuth {\tt tex} ({\sl not} {\tt etex}
or {\tt pdftex}) using {\sl only} the plain \TeX\ format is not
recommended.
The {\sl \TeX book}, for example, is set with the {\tt manmac}
macros to avoid tedious repetitions of low level commands
and to achieve a consistent layout.
The intention of the {\sl BookMacros} is to provide the minimum set
of macros and tools to create a book document using traditional
Knuth {\tt tex} with the plain \TeX\ format.
The goal had been to use as much as possible from plain \TeX\
directly and add only as little as necessary to keep writing
a document manageable and consistent.
Of course an author is encouraged to create
macros for each style markup element of a document.
The following features are supported:
\BeginItem
\Item Automatic table of contents and list of figures and
tables generation
\Item A set of headlines
\Item Displays
\Item Item, enumeration, and description lists
\Item Table and figure captions
\Item Footnote numbering
\Item Cross references and index
\End
Still there is much work to do, but currently work on this
package is stopped.
Next steps would have been:
\BeginItem
\Item Page header, roman page numbers in front and back matter,
two-sided documents
\Item Additional fonts
\Item Using MakeIndex instead of the current solution
\Paragr (It is not hard to put the main features of MakeIndex
into a script---but then why not {\sl use} MakeIndex \dots)
\Item Use of {\tt\RS pdfximage} instead of {\tt\RS epsfbox}
when {\tt pdftex} instead of {\tt tex} is used
\End
\Chapt{Elements to by taken from plain \TeX}
Some features for typesetting a document are already supported
by plain \TeX\ in a form that allows direct use in a document.
In short everything that is not mentioned in later chapters
needs to be taken from plain \TeX.
Please refer to \cite{knuth:ct:a} for a thorough description of plain \TeX.
\Sec{Title page}
Since the style of the initial pages of a book may vary significantly
there is no macro support for everything before the table of contents.
\Sec{Type area}
The default plain \TeX\ type area seems to be based on the letter format.
To achieve other formats the dimensions {\tt\RS hsize} and/or
{\tt\RS vsize} needs to be adjusted.
This document uses
\BeginDisp\tt
\RS advance\RS hsize by-6mm\RS advance\RS vsize by18mm
\End
for format A4.
This must be done {\sl before\/} the macro package is included since
it stores these values.
\Sec{Special characters}
\settabs\+\hskip4.5em&\cr
\+&{\tt\RS thinspace} (distance:
 \hbox{\vrule height 1.5ex depth 0ex
 \thinspace \vrule height 1.5ex depth 0ex}\thinspace)\cr
\+\`{}&{\tt\RS\`{}\lC\rC}\cr
\+\'{}&{\tt\RS\AQ\lC\rC}\cr
\+\~{}&{\tt\RS\~{}\lC\rC}\cr
\+$\bullet$&{\tt\$\RS bullet\$}\cr
\+\dag&{\tt\RS dag}\cr
\+\ddag&{\tt\RS ddag}\cr
\+\S&{\tt\RS S}\cr
\+\P&{\tt\RS P}\cr
\+\dots&{\tt\RS dots}\cr
\+``\SlNTS{text}''&{\tt\`{}\`{}\SlNTS{text}\AQ\AQ}\cr
\+{\it\$}&{\tt\lC\RS it\RS\$\rC}\cr
\+\$&{\tt\RS\$}\cr
\+\%&{\tt\RS\%}\cr
\+\&&{\tt\RS\&}\cr
\+\#&{\tt\RS\#}\cr
\+\_&{\tt\RS\_}\cr
\+$\langle$&{\tt\$\RS langle\$}\cr
\+$\rangle$&{\tt\$\RS rangle\$}\cr
\+$\{$&{\tt\$\RS\lC\$}\cr
\+$\}$&{\tt\$\RS\rC\$}\cr
\+$|$&{\tt\$|\$}\cr
\+\lq&\vtop{\advance\hsize by-4em
 {\tt\RS lq} (Left single quote.
 Can be used for a ligature with itself
 to produce a left double quote.)
}\cr
\+\rq&{\tt\RS rq}\cr
\+\"a&{\tt\RS "a} (In general an {\sl umlaut} is generated
 by preceding a letter with {\tt\RS "})\cr
\+\ss&{\tt\char`\\ss}\cr
\USec{Only in {\tt\RS tt} available special characters}
\settabs\+\hskip2em&\cr
{\tt
 \+<&<\cr
 \+>&>\cr
 \+|&|\cr
 \+"&"\cr
 \+\char`\ &\RS char\`{}\RS\char`\ \ {\rm(visible space)}\cr
}% \tt
\Sec{Tables}
Tables needs to set as described in the \TeX book, unfortunately.
\Ref{Chapt:TblCapt}
Only table captions are supported by this package, please refer to
chapter \RefHdNr\ (\RefHd)\RefPg.
\Sec{Keeps}
Keeps are achieved by enclosing something with
{\tt\RS vbox\lC}\dots{\tt\rC}.
\Sec{Floats}
Floats are created with
\BeginDisp\tt
\RS topinsert\SlNTS{vertical mode material}\RS endinsert
\End
(\cite[p. 115]{knuth:ct:a}).
If {\tt\RS midinsert} is used instead of {\tt\RS topinsert},
the float is placed at the current vertical position if possible
(\cite[p. 116]{knuth:ct:a}).
Else it has the same effect as {\tt\RS topinsert}.
\Sec{Minipages}
\hbox{%
 \vtop{\hsize=.5\hsize\advance\hsize by-2em
  If a narrow text area is needed (e.\thinspace g. in a table)
  it can be produced with {\tt\RS vtop} and a local change of
  {\tt\RS hsize}.
  If two such text areas should be placed side by side,
  they can be put in a {\tt\RS hbox} separated a {\tt\RS hskip}:
 }%
 \hskip1.5em%
 \vtop{\hsize=.5\hsize\advance\hsize by.5em
  \obeylines\tt\parindent=0sp
  \RS hbox\lC\%
  \ \RS vtop\lC\RS hsize=.5\RS hsize\RS advance\RS hsize by-.5em
  \ \ {\rm$\langle${\sl text \dots\/}$\rangle$}\rC\%
  \ \RS hskip1em\%
  \ \RS vtop\lC\RS hsize=.5\RS hsize\RS advance\RS hsize by-.5em
  \ \ {\rm$\langle${\sl text \dots\/}$\rangle$}\rC\%
  \rC
 }%
}
\Sec{Bibliography}
The plain \TeX\ macro package {\tt btxmac} provides the functionality
to produce a bibliography with a {\sl BibTeX} database.
The documentation of {\tt btxmac} is contained in it's source
code file and is similar to the usage in \LaTeX.
To use the package it needs to be included with
\BeginDisp\tt
\RS input btxmac
\End
at begin of the document.
A reference from the database is cited with
\BeginDisp\tt
\RS cite[\SlNTS{optional text}]\lC\SlNTS{database reference}\rC
\End
e.\thinspace g.
\BeginDisp\tt
\RS cite[p.\ 116]\lC knuth:ct:a\rC
\End
The bibliography is included with
\BeginDisp\tt
\RS nocite\lC *\rC
\RS bibliographystyle\lC plain\rC
\RS bibliography\lC\SlNTS{database name}\rC
\End
The generated {\tt aux} file needs to be processed
with the tool {\tt bibtex}.
\Chapt{Creating new documents}%
For each document a directory must be created first:
\BeginDisp\tt
mkdir -p \SlNTS{directory path}
cd \SlNTS{directory path}
\End
Then a symlink to the BookMacro directory is created:
\BeginDisp\tt
ln -s \SlNTS{path to BookMacro directory}
\End
and a symlink to the file {\tt configure} inside the {\tt BookMacro}
link is created:
\BeginDisp\tt
ln -s BookMacro/configure
\End
The {\sl makefile} assumes that the file names without extension of the input
and output file are the same.
The default input file name is {\tt doc.tex}, the default output name
is {\tt doc.dvi}.
A different name can be configured by creating a file
{\tt cfg.mk}\index{{\tt cfg.mk}}
which contains the line
\BeginDisp\tt
NAM = \SlNTS{name}
\End
Now a makefile needs to be generated with `{\tt./configure}'.
The document is then generated with `{\tt make}'.
`{\tt make clean}' removes all generated files except the makefile and
the output document.
With the target {\tt realclean}
really all generated files are removed.
\medskip\noindent
If a specific hyphenation language should be used,
some more settings need to be done in {\tt cfg.mk}.
\BeginDisp\tt
FMT = bplain\index{{\tt FMT}}\index{{\tt bplain} format}
\End
sets the required {\tt bplain} format.
\BeginDisp\tt
TEX = etex \it\% This is the default when setting FMT=bplain
\End
sets the tool (instead of traditional Knuth {\tt tex})
which is used for other hyphenation patterns.
\BeginDisp\tt
TEXFLAGS = -fmt \$(FMT) \it\% This is the default when setting FMT=bplain
\End
specifies the necessary tool options.
\medskip\noindent
The language is selected with the macro
\BeginDisp\tt
\RS Lang\lC\SlNTS{language}\rC
\End
e.\thinspace g.
\BeginDisp\tt
\RS Lang\lC ngerman\rC
\End
\USec{Document structure}%
At the beginning the macros needs to be included:
\BeginDisp\tt
\RS input BookMacros/BookMacros
\End
If cross references are used, it's data needs to be included next:
\BeginDisp\tt
\RS input xref
\End
Normally page breaks are done before chapter headings.
To avoid these page breaks set the page eject level higher than 2:
\BeginDisp\tt
\RS def\RS EjectLevel\lC1\rC\index{{\tt\RS EjectLevel}}
\End
The title page has to be created manually.
A simple approach (for a title without title page) is to type
\BeginDisp\tt
\RS centerline\lC\RS seventeenss \SlNTS{title}\rC
\End
The table of contents is inserted by the command
\BeginDisp\tt
\RS Toc\index{{\tt\RS Toc}}
\End
Chapters are added with {\tt\RS Chapt}, sections with {\tt\RS Sec},
paragraph headings with {\tt\RS Par}:
\BeginDisp\tt
\RS Chapt\lC\SlNTS{title}\rC
\RS Sec\lC\SlNTS{title}\rC
\RS Par\lC\SlNTS{title}\rC
\End
If an index is used it is included with the following statements:
\BeginDisp\tt
\RS UChapt\lC Index\rC
\RS input idx
\End
The final statement of each document is
\BeginDisp\tt
\RS bye
\End
\Chapt{Special characters}\index{special characters}%
\settabs\+\hskip5em&\cr
	\index{{\tt\RS Bq} (low double comma quotation mark)}
	\index{low double comma quotation mark (`\Bq')}
	\index{quotation mark}
\+\Bq\SlNTS{text}``&{\tt\RS Bq \SlNTS{text}\`{}\`{}}
(german quotation marks)\cr
\USec{Only in {\tt\RS tt} available special characters}
\settabs\+\hskip2.5em&\cr
{\tt
	\index{{\tt\RS lC} (left curly bracket)}
	\index{left curly bracket (`{\tt\lC}')}
\+\lC&\RS lC {\rm(left curly bracket)}\cr
	\index{{\tt\RS rC} (right curly bracket)}
	\index{right curly bracket (`{\tt\rC}')}
\+\rC&\RS rC {\rm(right curly bracket)}\cr
	\index{{\tt\RS AQ} (apostrophe quote)}
	\index{apostrophe quote (`{\tt\AQ}')}
\+\AQ&\RS AQ {\rm(apostrophe quote)}\cr
	\index{{\tt\RS RS} (reverse slash)}
	\index{reverse slash (`{\tt\RS}')}
\+\RS&\RS RS {\rm(reverse slash)}\cr
}% \tt
\Chapt{Fonts}\index{fonts}%
Table~\Ref{tblFonts}\RefNr\RefPg\ shows additional fonts available in the
BookMacros.
\midinsert
\newdimen\Hsize\Hsize=7cm
\newdimen\Lsize\Lsize=2.3cm
\newdimen\Msize\Msize=\Hsize
\newdimen\Rsize\Rsize=1cm
\advance\Msize by-\Lsize
\advance\Msize by-\Rsize
\advance\Msize by-.5cm % 2*gap
\def\TblRow#1#2#3{%
 \medskip
 \hbox to\Hsize{%
  \hbox to\Lsize{\tt\RS#1\hfil}\hfil
  \vtop{\hsize=\Msize \noindent#2}\hfil
  \hbox to\Rsize{\hfil#3\hfil}%
 }%
}%
\line{\hfil\hbox to\Hsize{%
 \vbox{%
  \hbox to\Hsize{\hfil\TblCapt{Additional fonts}\hfil}%
  \Label{tblFonts}%
  \smallskip
  \hrule width\Hsize
  \smallskip
  \hbox to\Hsize{%
   \hbox to\Lsize{\hfil\bf Name\hfil}\hfil
   \hbox to\Msize{\hfil\bf Description\hfil}\hfil
   \hbox to\Rsize{\hfil\bf Size\hfil}%
  }%
  \TblRow{seventeenss}{{\seventeenss Sans serif 17}}{17}%
   \index{{\tt\RS seventeenss}}\index{Sans serif}%
  \TblRow{twelvess}{{\twelvess Sans serif 12}}{12}%
   \index{{\tt\RS twelvess}}%
  \TblRow{tenssbx}{{\tenssbx Bold sans serif 10}}{10}%
   \index{{\tt\RS tenssbx}}%
  \smallskip
  \hrule width\Hsize}}\hfil}%
\endinsert
\Chapt{Headings and Table of Contents}\index{heading}%
\USec{Chapter headings}\index{chapter}%
\BeginDesc
\Item{\tt\RS FChapt\lC\SlNTS{title}\rC}
\index{{\tt\RS FChapt}}\index{unnumbered chapter}%
Unnumbered chapter without TOC entry intended for the preamble of a document.
This macro is used by {\tt\RS Toc}.
\Item{\tt\RS Chapt\lC\SlNTS{title}\rC}
\index{{\tt\RS Chapt}}%
Normal chapter heading.
The chapter heading font is defined by the macro
{\tt\RS ChaptFont}.
\Item{\tt\RS UChapt\lC\SlNTS{title}\rC}
\index{{\tt\RS UChapt}}%
Unnumbered chapter with TOC entry intended for the appendix of a document.
\End
If a page break occurs before a heading is configured with register
{\tt\RS EjectLevel}\index{{\tt\RS EjectLevel}}.
Page breaks occure if the eject level is higher than the heading level.
Since the default eject level is 2 page breaks normally are done before
chapters but not before sections.
\USec{Section headings}\index{section}%
\BeginDesc
\Item{\tt\RS Sec\lC\SlNTS{title}\rC}
\index{{\tt\RS Sec}}%
Normal section heading.
The section heading font is defined by the macro
{\tt\RS SecFont}.
\Item{\tt\RS USec\lC\SlNTS{title}\rC}
\index{{\tt\RS USec}}\index{unnumbered section}%
Unnumbered section without TOC entry.
\End
\USec{Paragraph heading}\index{paragraph heading}%
\BeginDesc
\Item{\tt\RS Par\lC\SlNTS{title}\rC}
\index{{\tt\RS Par}}%
Paragraph heading.
\End
\USec{Table of Contents}\index{table of contents}%
The table of contents is output with the
\BeginDisp\tt
\RS Toc\index{{\tt\RS Toc}}
\End
command.
It precedes a heading with the default name ``\TocNam''.
This name can be changed by redefining {\tt\RS TocNam}.
\USec{List of tables}\index{list of tables}%
A list of tables can be output with the
\BeginDisp\tt
\RS Lot\index{{\tt\RS Lot}}
\End
command.
It precedes a heading with the default name ``\LotNam''.
This name can be changed by redefining {\tt\RS LotNam}.
\USec{List of figures}\index{list of figures}%
A list of figures can be output with the
\BeginDisp\tt
\RS Lof\index{{\tt\RS Lof}}
\End
command.
It precedes a heading with the default name ``\LofNam''.
This name can be changed by redefining {\tt\RS LofNam}.
\Chapt{Displays}\index{display}%
Displays draw text with one line of output for each input line
indented on the left side.
Display neither have the keep\index{keep!I} nor the float\index{float!I}
function.
For keeps {\tt\RS vbox} is required, for floats e.g.
{\tt\RS midinsert} can be used.
Input
\BeginDisp\tt
\vbox{%
\RS BeginDisp\index{{\tt\RS BeginDisp}}
\RS vbox\lC\% only required for keep function
First line
second line
\rC\%
\RS End\index{{\tt\RS End}}
}% \vbox
\End
yields
\BeginDisp
First line
second line
\End
If the display should be set in typewriter text {\tt\RS tt} is
simply appended to {\tt\RS BeginDisp}
({\tt\RS BeginDisp} and {\tt\RS End} form a group):
\BeginDisp\tt
\RS BeginDisp\RS tt
{\rm\dots}
\RS End
\End
Displays and the following text don't have an indentation of the first line.
\BeginNarrow\index{indented text blocks}%
For indentation of adjusted text at the left and right side the macros
{\tt\RS BeginNarrow} and {\tt\RS End} are available:
\BeginDisp\tt
\RS BeginNarrow\index{{\tt\RS BeginNarrow}}
{\rm\dots}
\RS End
\End
{\tt\RS BeginNarrow} and {\tt\RS End} form a group.
Font changes may simply be appended to {\tt\RS BeginNarrow}.
\End
Indented text blocks and the following text don't have an indentation
of the first line.
\Chapt{Lists}\index{lists}%
\Sec{Item Lists}\index{item lists}%
An item list has the following structure:
\BeginDisp\tt
\RS BeginItem\index{{\tt\RS BeginItem}}
\RS Item\index{{\tt\RS Item}} \SlNTS{text}
\RS Paragr\index{{\tt\RS Paragr}} \SlNTS{text}
\RS Item \SlNTS{text}
\RS End
\End
{\tt\RS Paragr} starts a new paragraph within an item.
Example:\par\noindent
\hbox{%
 \vtop{\hsize=.3\hsize\advance\hsize by-1em\noindent
  \BeginItem
  \Item Item 1
   \BeginItem
   \Item Item 1.1
    \BeginItem
    \Item Item 1.1.1
    \Item Item 1.1.2
    \End
   \Item Item 1.2
   \Item Item 1.3
   \End
  \Item Item 2
  \Paragr Paragr
  \Item Item 3
  \End
 }%
 \hskip2em%
 \vtop{\hsize=.7\hsize\advance\hsize by-1em\noindent
  \BeginDisp\tt
   \RS BeginItem
   \RS Item Item 1
   \ \RS BeginItem
   \ \RS Item Item 1.1
   \ \ \RS BeginItem
   \ \ \RS Item Item 1.1.1
   \ \ \RS Item Item 1.1.2
   \ \ \RS End
   \ \RS Item Item 1.2
   \ \RS End
   \RS Item Item 2
   \RS Paragr Paragr
   \RS Item Item 3
   \RS End
  \End
 }%
}
\Sec{Enum Lists}\index{enum lists}%
An enum list has the following structure:
\BeginDisp\tt
\RS BeginEnum\index{{\tt\RS BeginEnum}}
\RS Item\index{{\tt\RS Item}} \SlNTS{text}
\RS Paragr\index{{\tt\RS Paragr}} \SlNTS{text}
\RS Item \SlNTS{text}
\RS End
\End
{\tt\RS Paragr} starts a new paragraph within an item.
Example:\par\noindent
\hbox{%
 \vtop{\hsize=.3\hsize\advance\hsize by-1em\noindent
  \BeginEnum
  \Item Item 1
   \BeginEnum
   \Item Item 1.1
    \BeginEnum
    \Item Item 1.1.1
    \Item Item 1.1.2
    \End
   \Item Item 1.2
   \Item Item 1.3
   \End
  \Item Item 2
  \Paragr Paragr
  \Item Item 3
  \End
 }%
 \hskip2em%
 \vtop{\hsize=.7\hsize\advance\hsize by-1em\noindent
  \BeginDisp\tt
   \RS BeginEnum
   \RS Item Item 1
   \ \RS BeginEnum
   \ \RS Item Item 1.1
   \ \ \RS BeginEnum
   \ \ \RS Item Item 1.1.1
   \ \ \RS Item Item 1.1.2
   \ \ \RS End
   \ \RS Item Item 1.2
   \ \RS End
   \RS Item Item 2
   \RS Paragr Paragr
   \RS Item Item 3
   \RS End
  \End
 }%
}
\Sec{Description Lists}\index{description lists}%
A description list has the following structure:\par\noindent
\hbox{%
 \vtop{\hsize=.35\hsize\advance\hsize by-1em\vskip-1ex\noindent
  \BeginDesc
  \Item{\SlNTS{First tag}}\SlNTS{Descritpion text one}
  \Item{\SlNTS{Second tag}}\SlNTS{Description text two}
  \Paragr \SlNTS{Further text}
  \End
 }%
 \hskip2em%
 \vtop{\hsize=.65\hsize\advance\hsize by-1em\noindent
  \BeginDisp\tt
  \RS BeginDesc\index{{\tt\RS BeginDesc}}
  \RS Item\lC\SlNTS{First tag}\rC\
  \SlNTS{Descritpion text one}
  	\index{{\tt\RS Item}}
  \RS Item\lC\SlNTS{Second tag}\rC\
  \SlNTS{Description text two}
  \RS Paragr\index{{\tt\RS Paragr}} \SlNTS{Further text}
  \RS End
  \End
 }%
}\smallbreak\noindent
Space after `{\tt\rC}' does not influence the output.
If a new paragraph should be started within an item, {\tt\RS Paragr}
is used.
\Chapt{Table captions}\Label{Chapt:TblCapt}
Table captions are set with the {\tt\RS TblCapt} command:
\BeginDisp\tt
\RS TblCapt[\SlNTS{optional short caption}]\lC\SlNTS{caption text}\rC
\End
If an \SlNTS{optional short caption} is given, it is used instead of
\SlNTS{caption text} in the list of tables.
To support flexible placement schemes (centered above the table,
on top right to the table, etc.), the user is responsible for the
caption placement.
In simple cases something like
\BeginDisp\tt
\RS hbox to \SlNTS{table\
width}\lC\RS hfil\RS TblCapt\lC\SlNTS{caption}\rC\RS hfil\rC
\RS smallskip
\End
above the table should be sufficient.
The caption is preceded by ``{\bf\TblNam}'' (followed by the
table number).
This text can be changed be redefining {\tt\RS TblNam}.
\Chapt{EPS file inclusion}
To enable EPS inclusion the {\tt epsf} macros need to be included
by placing
\BeginDisp\tt
\RS input epsf
\End
at the begin of the document.
The command to include the picture is
\BeginDisp\tt
\RS EpsfBox\lC\SlNTS{file name}\rC
\End
e.\thinspace g.
\BeginDisp\tt
\RS EpsfBox\lC ctanlion.eps\rC
\End
The macro sets {\tt\RS EpsfXsize} to the width of the picture.
So if text should be placed beside the picture it can be done
using the {\tt\RS FigCapt} macro with
\BeginDisp\tt
\RS vbox\lC\RS advance\RS hsize by-\RS EpsfXsize\RS noindent
\ \RS FigCapt[\SlNTS{optional short caption}]\lC\SlNTS{caption text}\rC\rC
\End
If an \SlNTS{optional short caption} is given, it is used instead of
\SlNTS{caption text} in the list of figures.
The caption is preceded by ``{\bf\FigNam}'' (followed by the
figure number).
This text can be changed be redefining {\tt\RS FigNam}.
Example:\smallskip\nobreak\noindent
\EpsfBox{ctanlion.eps}
\vbox{\advance\hsize by-\EpsfXsize\noindent
 \FigCapt[Unscaled EPS image]{Text that follows normally appears
  on the right with the same baseline}}
\medbreak\noindent
Pictures can be scaled by setting {\tt\RS epsfxsize=}\RmNTS{dimen}
and/or {\tt\RS epsfysize=}\RmNTS{dimen}, e.\thinspace g.
\BeginDisp\tt
\RS epsfxsize=5cm
\RS centerline\lC\RS EpsfBox\lC ctanlion.eps\rC\RS FigCapt\lC Scaled\
EPS image\rC\rC
\End
produces:\smallskip\nobreak\noindent
\epsfxsize=5cm
\centerline{\EpsfBox{ctanlion.eps}\FigCapt{Scaled EPS image}}
\Chapt{Footnotes}
Plain \TeX\ provides a ready to use solution for
footnotes\footnote{*}{Like this which is produced with
{\tt\RS footnote\lC*\rC\lC\SlNTS{text}\rC},
see \cite[p. 116]{knuth:ct:a}.}.
To have automatic footnote numbering
\BeginDisp\tt
\RS Footnote\lC\SlNTS{text}\rC
\End
can be used.
The footnote counter\Footnote{{\tt\RS FootnoteCtr}}
is reset when the macro package calls
{\tt\RS supereject} before a headline.\Footnote{Depends on the
{\tt\RS EjectLevel} setting.}
Further references to a previous footnote\FootnoteRef\ can be marked with
{\tt\RS FootnoteRef}.
{\tt\RS footnote} must not be used in an {\sl insert}.
{\tt\RS vfootnote}\Footnote{\cite[p. 117]{knuth:ct:a}} may be used instead.
\Chapt{Cross references}\index{cross references}%
A reference mark can be placed with
\BeginDisp\tt
\RS Label\lC\SlNTS{label}\rC
\End
at a heading, table, figure etc.
To reference a label the label data needs to be loaded with
\BeginDisp\tt
\RS Ref\lC\SlNTS{label}\rC
\End
After that the following macros are set:
\BeginDesc
\Item{\tt\RS RefPgNr} Reference page number.\index{{\tt\RS RefPgNr}}
\Item{\tt\RS RefNr} Number of table or figure.\index{{\tt\RS RefNr}}
\Item{\tt\RS RefHdNr} Number of heading.\index{{\tt\RS RefHdNr}}
\Item{\tt\RS RefHd} Title of the last heading before the label.
 \index{{\tt\RS RefHd}}%
\Item{\tt\RS RefPg} Outputs text:\index{{\tt\RS RefPg}}
 \BeginDesc
 \Item{\RmNTS{nothing}} if the label is on the
  current page,
 \Item{\rm`` on the previous page''} if the label is on the previous page,
 \Item{\rm`` on the following page''} if the label is on the next page, and
 \Item{\rm`` on page {\tt\RS RefPgNr}''} else.
 \End
\Paragr (Notice the leading space.)
\End
\Chapt{Index}\index{index}%
Words are added to the index with the {\tt\RS Idx} macro:\index{{\tt\RS Idx}}
\BeginDisp\tt
\RS index\lC\SlNTS{word or symbol}\rC
\End
If the words to add belong to a main category, they are separated
by a `{\tt!}':
\BeginDisp\tt
\RS index\lC\SlNTS{main word}!\SlNTS{word or symbol}\rC
\End
A `{\tt!}' at the begin or end of the argument to {\tt\RS Idx} it is
recognized as itself.
If a `{\tt!}' should be placed inside other text it must be doubled.
More that two consecutive `{\tt!}' need to be separated by `{\tt\lC\rC}':
\BeginDisp\tt
\RS index\lC Special characters!\lC\rC!!\rC
\End
For special markup additional symbols can be added to the argument,
again separated by `{\tt!}' (see table~\Ref{tblIdxTypes}\RefNr\RefPg).
If such such a symbol sequence is actually meant to be added to the index
it needs to be doubled.
The index generator removes such sequences one time from the end of the
argument.
If it apears a second time it is included into the index.
Combinations of such markup symbols are possible:
\BeginDisp\tt
\RS index\lC \RS idx!(!B\rC
\End
\midinsert
\newdimen\Hsize\Hsize=.98\hsize
\newdimen\Lsize\Lsize=.200\Hsize
\newdimen\Msize\Msize=\Hsize
\newdimen\Rsize\Rsize=.282\Hsize
\advance\Msize by-\Lsize
\advance\Msize by-\Rsize
\advance\Msize by-.7cm % 2*gap
\def\TblRow#1#2#3{%
 \medskip
 \hbox to\Hsize{%
  \vtop{\hsize=\Lsize \noindent#1}\hfil
  \vtop{\hsize=\Msize \noindent#2}\hfil
  \vtop{\hsize=\Rsize \noindent#3}%
 }%
}%
\line{\hfil\hbox to\Hsize{%
 \vbox{%
  \hbox to\Hsize{\hfil\TblCapt{Types of index entries}\hfil}%
  \Label{tblIdxTypes}%
  \smallskip
  \hrule width\Hsize
  \smallskip
  \hbox to\Hsize{%
   \hbox to\Lsize{\hfil\bf Markup\hfil}\hfil
   \hbox to\Msize{\hfil\bf Description\hfil}\hfil
   \hbox to\Rsize{\bf Symbol\hfil}%
  }%
  \TblRow{Bold page number}{Text reference contains important
   information for the subject.}{`{\tt B}'}%
  \TblRow{Italic page number}{Text reference contains unimportant
   information for the subject (``{\sl also mentioned there} \dots'').}{%
   `{\tt I}'}%
  \TblRow{Range of page numbers}{Subject is descripted on these range
   of pages.}{Use `{\tt(}' for the first page and `{\tt)}' for the
   last page of the range.}%
  % TODO: "see" (no page number), "see also" (with `;' after page numbers)
  \smallskip
  \hrule width\Hsize}}\hfil}%
\endinsert
\ifSpec
\noindent
The argument to {\tt\RS Idx} is written to the file {\tt idx.txt} in the
form \SlNTS{argument}\SlNTS{space}\SlNTS{page number}.
\fi
\UChapt{Bibliography}
\nocite{*}
\bibliographystyle{plain}
\bibliography{local}
\vfil\supereject
\UChapt{Index}
\input idx
\bye
